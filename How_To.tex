\documentclass{article}

\usepackage[utf8]{inputenc}
\usepackage[T1]{fontenc}
\usepackage{lmodern}
\usepackage{ngerman}
\usepackage{url}
\usepackage{graphicx}


\pagestyle{empty}

\begin{document}
\title{ How To }
\author{ Benjamin Arnold \and Felix Hoeborn }
\date{ \today }
\maketitle
\newpage
\tableofcontents
\newpage
\setcounter{page}{1}
\pagestyle{plain}

\section{Top-Level}
Diese Sparche hat auf oberster Ebene 4 Bestandteile.

\subsection{Includes}
\texttt{include} 'Name der zu includierenden Datei' \texttt{;} \\

\subsection{Defines} \label{define}
Es können globale Variabelen definiert werden. \\
\texttt{<} \textit{type} \texttt{>} \textit{id}  \texttt{;} f"ur nicht initiatisierte Variabelen. \\
\texttt{<} \textit{type} \texttt{>} \textit{id} \texttt{=} \textit{term} \texttt{;} f"ur nicht initiatisierte Variabelen.

\subsection{Klassen} \label{struct}
\textit{id} \texttt{\{}
\begin{itemize}
\item[]{Innerhalb einer Klasse können Definitionen  {\small(siehe \ref{define})} stehen.}
\item[]{Konstruktoren können nach folgendem Muster definiert werden:\\
\texttt{cons} \textit{id} \texttt{(<} \textit{type} \texttt{>} \textit{id} {\small( , \texttt{<} \textit{type} \texttt{>} \textit{id} , ... )} \texttt{)}\\ 
\texttt{\{}\\
\textit{expr} {\tiny\ref{expr}} und/oder \textit{define} {\tiny\ref{define}}\\
\texttt{\}}}
\end{itemize}
\texttt{\}}

\subsection{Funktionsdefinition} \label{function}
Funktionen werden wie folgt definiert: \\
\texttt{<} \textit{type} \texttt{>} \textit{id}\texttt{(<} \textit{type} \texttt{>} \textit{id}  {\small( , \texttt{<} \textit{type} \texttt{>} \textit{id} , ... )} \texttt{)}\\
\texttt{\{}\\
\textit{expr} {\tiny\ref{expr}} und/oder \textit{define} {\tiny\ref{define}}\\
\texttt{\}}


\section{Low-Level}

\subsection{Expressions} \label{expr}
\subsubsection{Operationen}
Operationen haben immer den gleichen Aufbau.\\
\textit{term} {\tiny\ref{term}} \textit{Operant} \textit{expr} {\tiny\ref{expr}}\\
Operanten sind \texttt{+} , \texttt{-} , \texttt{*} , \texttt{/} , \texttt{\%} , \texttt{is} ,\texttt{smaller} und \texttt{bigger} 

\subsubsection{Funktionsaufrufe}
Es gibt 3 Arten von Funktionsaufrufen:
\begin{itemize}
\item[call]{\texttt{call} \textit{id}\texttt{(} \textit{id}  {\tiny( , \textit{id} , ... )} \texttt{)} }
\item[new]{\texttt{new} \textit{id}\texttt{(} \textit{id}  {\tiny( , \textit{id} , ... )} \texttt{)} }
\item[destroy]{\texttt{destroy} \textit{id}}
\end{itemize}

\subsubsection{Set}
\textit{id} \texttt{=} \textit{term} \texttt{;}

\subsubsection{Terme}
F"ur eine expr kann auch ein term{\tiny\ref{term}} eingesetzt werden.

\subsection{Terme} \label{term}
\begin{itemize}
\item[expr]{\texttt{(} expr \texttt{)}}
\item[id]{Zeichenketten mit großen und kleinen Buchstaben sowiet Sterne }
\item[number]{ganze Zahlen }
\item[null]{\texttt{null} }
\item[boolean]{\texttt{true} oder \texttt{false} }
\end{itemize}


\newpage


\end{document}